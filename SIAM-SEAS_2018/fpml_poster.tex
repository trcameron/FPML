\documentclass[orientation=portrait]{tikzposter}
\usepackage[utf8]{inputenc}
\usepackage{amsmath}

\geometry{paperwidth=24in,paperheight=36in}
\makeatletter
\setlength{\TP@visibletextwidth}{\textwidth-2\TP@innermargin}
\setlength{\TP@visibletextheight}{\textheight-2\TP@innermargin}
\makeatother

\title{FPML}
\author{Thomas R. Cameron}
\date{\today}
\institute{Davidson College}
 
\usepackage{blindtext}
\usepackage{comment}

\usetheme{Autum}
 
\begin{document}
 
\maketitle

%%%%%%%%%%%%%%%%%%%%%%%%%%%%%%%%%%%%%%%%
%								Abstract						%
%%%%%%%%%%%%%%%%%%%%%%%%%%%%%%%%%%%%%%%%
\block{Abstract}
{
We present a novel modification of Laguerre's method that results in a method for the concurrent approximation of all roots of a univariate polynomial. Our method has strong virtues including fourth-order convergence that is observed in practice and belonging to the class of embarrassingly parallel algorithms. A Fortran 90 implementation of our algorithm is available online and comparisons with several other software are provided to show the effectiveness of our approach.
}

%%%%%%%%%%%%%%%%%%%%%%%%%%%%%%%%%%%%%%%%
%								Introduction					%
%%%%%%%%%%%%%%%%%%%%%%%%%%%%%%%%%%%%%%%%
\begin{columns}
	\column{0.5}
	\block{Introduction}
	{
		Let $p(\lambda)$ be a polynomial of degree $m$ and denote by $(z_{1},\ldots,z_{m})$ the current approximations to the roots $r_{1},\ldots,r_{m}$ of $p(\lambda)$. The $j$th approximation is updated via
		\begin{equation}
		\hat{z}_{j}=z_{j}-\frac{m}{G_{j}\pm\sqrt{(m-1)(mH_{j}-G_{j}^{2})}},
		\end{equation}
		where 
		\begin{equation}
		G_{j}=\frac{p^{'}(z_{j})}{p(z_{j})}-\sum_{\substack{i=1\\i\neq j}}^{m}\frac{1}{(z_{j}-z_{i})}~\text{ and }~H_{j}=
		\end{equation}
	}
	
	\column{0.5}
	\block{Something else}{Here, \blindtext
	\vspace{4cm}}
	\note[
		targetoffsetx=-9cm, 
        		targetoffsety=-6.5cm, 
        		width=0.5\linewidth
		]
		{email\texttt{thcameron@davidson.edu}}
\end{columns}
 
\end{document}